%
% ============Instellingen:============
%
%\documentclass[12pt]{article}
\documentclass[12pt,a4paper]{article}
\special{papersize=210mm,297mm}
% Bovenstaande bevelen zorgen ervoor dat een bladzijde een echte A4 is,2 cm langer
\usepackage[top=2.5cm, bottom=3cm, left=2.5cm, right=2.5cm]{geometry}
% Bovenstaand toegevoegd bevel reduceert de marges rondom een blad.
% zie: https://en.wikibooks.org/wiki/LaTeX/Page_Layout#Page_size_issues
%\usepackage[english]{babel}
\usepackage[dutch]{babel}
% zie ook: https://nl.wikipedia.org/wiki/LaTeX
% Het pakket babel zorgt ervoor dat al de benamingen in het Nederlands zijn
\selectlanguage{dutch}
% Zorgt ervoor dat het Nederlands ook echt wordt gebruikt...
% Zie: https://groups.google.com/forum/#!topic/nl.comp.os.linux/l1M_VF8f8tc
\usepackage[utf8x]{inputenc}
% Zie http://www.faqoverflow.com/tex/13067.html: utf8x gebruikt ucs package en daar zit geen onderhoud meer op. Best utf8 gebruiken. Het gaat over het gebruik van speciale tekens... .
%\usepackage{amsmath}
% Zie: http://tex.stackexchange.com/questions/32100/what-does-each-ams-package-do
\usepackage{amsmath, amsthm, amssymb, amsfonts} 
\usepackage{graphicx}
% om figuren te kunnen opnemen in de tekst.
% Zie: https://en.wikibooks.org/wiki/LaTeX/Importing_Graphics
\usepackage[colorinlistoftodos]{todonotes}
%\colorbox{yellow}{TO DO / ex. reference}
% Deze kan zeker worden overgeslagen: noteren van todo's.
%http://tex.stackexchange.com/questions/9796/how-to-add-todo-notes
\usepackage{float}
% Floats dienen om vb. figuren en tabellen niet te splitsen over meerdere bladzijden.
% Zie: https://en.wikibooks.org/wiki/LaTeX/Floats,_Figures_and_Captions
\usepackage{caption}
%captions dienen voor bijschriften.
% Zie: https://en.wikibooks.org/wiki/LaTeX/Floats,_Figures_and_Captions
\usepackage{subcaption}
% Zie http://tex.stackexchange.com/questions/40460/how-to-put-subcaptions-below-subtables-and-increase-gap-between-subtables-with
% zie ook: https://nl.sharelatex.com/blog/2013/08/07/thesis-series-pt3.html
\usepackage{wrapfig}
% laat toe tekst rond de figuur te plaatsen.
% zie: https://nl.wikibooks.org/wiki/LaTeX/Figuren_en_drijvende_omgevingen
%...
%die 49 jaar werd.
%\begin{wrapfigure}{r}{0.5\textwidth}
%  \includegraphics[width=0.45\textwidth]{Gull}
%  \caption{Foto van een meeuw} 
%\end{wrapfigure}
%Meeuwen broeden in grote, drukke en lawaaierige kolonies
%...
\usepackage{framed}
% Een box is tekst die samenhoort op vb. 1 bladzijde.
% Zie https://en.wikibooks.org/wiki/LaTeX/Boxes
% met bijvoorbeeld
%\begin{framed}
%This is an easy way to box text within a document!
%\end{framed}
% kan je tekst zelf als 1 geheel definiëren.
\usepackage{stfloats}
% Dit is een interessante:
% Zie https://www.ctan.org/pkg/stfloats
% plaats van caption, baselineskip, ... 
\usepackage{pdflscape}
% Maakt dat je landscape kan werken.
% Zie: https://www.ctan.org/pkg/pdflscape
\usepackage{csquotes}
% Deze gaat over het automatisch vervangen van quotes zodat je 1 quote ziet in je tekst.
% Zie: https://www.ctan.org/pkg/csquotes
%
% Meerdere kolommen:
% Zie ook: https://nl.sharelatex.com/learn/Multiple_columns
\usepackage{multicol}
\usepackage{color}
% Laat toe om met kleur te werken, vb tekst in een kleur, tekst markeren in een kleur, etc.
% Zie: https://en.wikibooks.org/wiki/LaTeX/Colors
% \colorbox{declared-color}{TO DO / ...}
% \textcolor{declared-color}{text}
% Tabellen naast elkaar: subtables:
% Zie : http://tex.stackexchange.com/questions/36077/argument-in-beginsubtable
% Figuren naast elkaar: subfigures:
% Zie: https://nl.sharelatex.com/learn/Positioning_images_and_tables
%\usepackage{fontspec}
%\setmainfont{Calibri.ttf}
% Op deze manier kan je een ander lettertype gebruiken.
% Tekst in vet, schuin of onderlijnd:
% https://nl.sharelatex.com/learn/Bold,_italics_and_underlining
% Some of the \textbf{greatest} discoveries in \underline{science} were made by \textit{accident}.
%\parskip = \baselineskip
% Bevel verplaatst na table of content.
% Vorig bevel: voeg een extra blanco lijn toe na elke paragraaf
% Zie: http://tex.stackexchange.com/questions/74170/have-new-line-between-paragraphs-no-indentation
\setlength{\parindent}{0em}
% Geen identatie
% Zie: https://nl.sharelatex.com/learn/Paragraph_formatting#Paragraph_Indentation
% Opsommingen:
% Zie: https://en.wikibooks.org/wiki/LaTeX/List_Structures
% Zie: https://nl.sharelatex.com/learn/Lists
%Soorten lijsten:
%- itemize for a bullet list
%- enumerate for an enumerated list and
%- description for a descriptive list.
%
%\usepackage{enumitem}
%
%Voorbeelden:
%\begin{itemize}  
%\item The first item 
%\item The second item 
%\item The third item
%\end{itemize}
%of:
%\begin{itemize}[noitemsep] %[noitemsep]: bolletjes dichter bij elkaar
%\item The first item
%\item The second item
%\item The third item
%\end{itemize}
%en:
%\begin{enumerate}
%\item The first item
%\begin{enumerate}
%\item Nested item 1
%\item Nested item 2
%\end{enumerate}
%\item The second item
%\item The third etc \ldots
%\end{enumerate}
%en:
%\usepackage[inline]{enumitem}
%\usepackage{xcolor}
%\begin{document}
% Coco likes fruit. Her favorites are:
%\begin{enumerate*}[label={\alph*)},font={\color{red!50!black}\bfseries}]
%\item bananas
%\item apples
%\item oranges and
%\item lemons.
%\end{enumerate*}
% Referenties leggen:
%\label{referentiewoord}
%\ref{referentiewoord}
%\pageref{referentiewoord}
% https://nl.wikibooks.org/wiki/LaTeX/Labels_en_Referenties
\captionsetup[figure]{name=Figuur} 
% Om Nederlandse figuurnamen te hebben
\captionsetup[table]{name=Tabel}
%\captionsetup{font=footnotesize} toegevoegd om de bijschriften ook in een kleiner lettertype te hebben.
\captionsetup{font=footnotesize}
% http://tex.stackexchange.com/questions/82993/how-to-change-the-name-of-document-elements-like-figure-contents-bibliogr
\addto\captionsenglish{% Replace "english" with the language you use
  \renewcommand{\contentsname}%
    {Inhoudstafel}%
}

%Bruikbare lettertypes zijn:
%\Huge
%\huge
%\LARGE
%\Large
%\large
%\normalsize
%\small
%\footnotesize
%\scriptsize :extra klein voor brede tabellen in landscape 
%\tiny


voor de referenties:

%@article{greenwade93,
%    author  = "George D. Greenwade",
%    title   = "The {C}omprehensive {T}ex {A}rchive {N}etwork ({CTAN})",
%    year    = "1993",
%    journal = "TUGBoat",
%    volume  = "14",
%    number  = "3",
%    pages   = "342--351"
%},

%@book{goossens93,
%    author    = "Michel Goossens and Frank Mittelbach and Alexander Samarin",
%    title     = "The LaTeX Companion",
%    year      = "1993",
%    publisher = "Addison-Wesley",
%    address   = "Reading, Massachusetts"
%},

%@misc{website:fermentas-lambda,
%      author = "Fermentas Inc.",
%      title = "Phage Lambda: description \& restriction map",
%      month = "November",
%      year = "2008",
%      url = "http://www.fermentas.com/techinfo/nucleicacids/maplambda.htm"
%},

%@online{WinNT,
%  author = {MultiMedia LLC},
%  title = {{MS Windows NT} Kernel Description},
%  year = 1999,
%  url = {http://web.archive.org/web/20080207010024/http://www.808multimedia.com/winnt/kernel.htm},
%  urldate = {2010-09-30}
%}

%misc{website:DSNH,
%      author = "Martin D.",
%      title = "Deep Space Network history",
%      month = "22 juli",
%      year = "2005",
%      note = "Toegang: 5 oktober 2013",
%      url = "https://deepspace.jpl.nasa.gov/dsn/history/"
%},