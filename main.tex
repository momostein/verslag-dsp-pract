%%%%%%%%%%%%%%%%%%%%%%%%%%%%%%%%%%%%%%%%%
% University Assignment Title Page
% LaTeX Template
% Version 1.0 (27/12/12)
%
% This template has been downloaded from:
% http://www.LaTeXTemplates.com
%
% Original author:
% WikiBooks (http://en.wikibooks.org/wiki/LaTeX/Title_Creation)
%
% License:
% CC BY-NC-SA 3.0 (http://creativecommons.org/licenses/by-nc-sa/3.0/)
%
%
% ============Instellingen:============
%
%\documentclass[12pt]{article}
\documentclass[12pt,a4paper]{article}
\special{papersize=210mm,297mm}
\usepackage[top=2.5cm, bottom=3cm, left=2.5cm, right=2.5cm]{geometry}
%\usepackage[english]{babel}
\usepackage[main=dutch,english]{babel}
\selectlanguage{dutch}
\usepackage[utf8x]{inputenc}
%\usepackage{amsmath}
\usepackage{amsmath, amsthm, amssymb, amsfonts}
\usepackage{graphicx}
\usepackage{float}
\usepackage{caption}
\usepackage{subcaption}
\usepackage{wrapfig}
\usepackage{framed}
\usepackage{stfloats}
\usepackage{pdflscape}
% Maakt dat je landscape kan werken.
% Zie: https://www.ctan.org/pkg/pdflscape
\usepackage{csquotes}
\usepackage{multicol}
\usepackage[colorinlistoftodos]{todonotes}
\usepackage{color}
% \colorbox{declared-color}{TO DO / ...}
% \textcolor{declared-color}{text}
%\usepackage{fontspec}
%\setmainfont{Calibri.ttf}

%\parskip = \baselineskip
% Bevel verplaatst na table of content.
% Vorig bevel: voeg een extra blanco lijn toe na elke paragraaf

\setlength{\parindent}{0em}
\usepackage{enumitem}

\captionsetup[figure]{name=Figuur}
% Om Nederlandse figuurnamen te hebben
\captionsetup[table]{name=Tabel}
%\captionsetup{font=footnotesize} toegevoegd om de bijschriften ook in een kleiner lettertype te hebben.
\captionsetup{font=footnotesize}

\addto\captionsenglish{% Replace "english" with the language you use
  \renewcommand{\contentsname}%
    {Inhoudstafel}%
}

%
% ============ BEGIN VAN HET DOCUMENT ============
%
%
\begin{document}

%Titelpagina....
\input{titelblad.tex}

\small
\tableofcontents
\newpage
\listoffigures
\newpage
\listoftables
% Volgend bevel staat pas hier, want anders staat er een blanco lijn na elke lijn in de inhoudstafel en de lijsten van figuren en tabellen:
\parskip = \baselineskip
\newpage
\Large
\textbf{Lijst van afkortingen}
\newline
\small
\begin{table}[ht]
% https://nl.wikibooks.org/wiki/LaTeX/Tabellen
\small
\begin{tabular}{ll} 
AM &Amplitudemodulatie (Amplitude Modulation) \\
DSN &Deep Space Network\\
FM &Frequentiemodulatie (Frequency Modulation) \\
NASA &National Aeronautics and Space Administration \\
\end{tabular}
\end{table}

\newpage

%\begin{multicols}{2}
\section{Inleiding}

Nu machine learning steeds meer toepassingen krijgt, gaat men ook veel meer machine learning modellen gebruiken in de productieomgeving.
Traditioneel werden die modellen toegepast op centrale servers of gedistribueerde cloud servers. Hierdoor moet er een grote hoeveelheid data verstuurd worden over het internet.
Dat is echter niet zo schaalbaar als men op grote schaal met heel veel verschillende toestellen data gaat opmeten. % TODO: Waarom niet schaalbaar?

\begin{figure}[ht]
	\centering
	\includegraphics[width=0.9\textwidth]{figuren/iisedgecomputing.jpg}
	\caption{Edge vs. Cloud computing}
	\cite{flir-edge-computing}
	\label{fig:edge-vs-cloud}
\end{figure}

Er is gelukkig een grote 'paradigm shift' aan de gang waarbij men steeds meer machine learning gaat uitvoeren in embedded devices.
Dat is mogelijk omdat de embedded devices steeds sneller en beter worden in het verwerken van data.
Op die manier kunnen wij het machine learning model dichter bij de sensor plaatsen waardoor wij het benodigde netwerkverkeer minimaliseren.
Ook de latency of vertraging tussen opname en actie is veel kleiner. \cite{flir-edge-computing}

De ruwe data van de sensoren kan soms ook sensitieve informatie zoals gezichten of nummerplaten bevatten. Omdat we de data on-edge verwerken hoeven we die niet naar de cloud te sturen waar men ze kan onderscheppen. \cite{flir-edge-computing}
Het is echter niet zo simpel om de machine learning code, die geoptimaliseerd is voor multi-core servers, zomaar over te zetten naar de kleine, vaak single-core, microprocessors van die embedded devices.
% In dit verslag leg ik het proces om een machine learning model getraind in python, over te zetten naar C code voor een microcontroller.

\subsection{Doel van het project}

Het doel van dit project is met een Convolutional Neural Netwerk (CNN) geluidsevents te classificeren.
Die geluidsevents kunnen bijvoorbeeld spraak, een handenklap, gefluit of achtergrondgeluid zijn.
Dat is relatief simpel uit te voeren op een normale computer, maar het is dus niet praktisch om alle geluidsdata van meerdere meetpunten centraal te verwerken in real time.
Ook is het niet kost-effectief om overal aparte computers te plaatsen. In dit project gaan we dus het getraind CNN op een microcontroller uitvoeren om geluid van de ingebouwde microfoon te classificeren.

\section{Log-Mel feature extraction}

Het is mogelijk om een CNN te trainen op de onverwerkte geluidsdata.
Er zijn echter veel grotere modellen voor nodig, tot 34 lagen diep, en zal daarom ook sneller vatbaar voor overfitting. \cite{IEEE:very-deep-cnn-raw-waveforms}
Daarom is er toch een feature extraction stap gemaakt om de onverwerkte geluidsdata om te rekenen naar een Log-Mel-spectrogram.
Dit Log-Mel-spectrogram is eigenlijk een gewoon spectrogram waarvan we de frequentiebanden een beetje vervormd hebben zodat ze beter overeenkomen met de menselijke perceptie van geluid. \cite{enwiki:Mel-freq-cepstrum}

\begin{no-awa} % Zorgt ervoor dat de latex-2-awa parser deze environment negeert
De  Log-Mel feature extraction gebeurt in deze 5 stappen:

\begin{enumerate}
	\item DC-Normalisation
	      \begin{itemize}
		      \item Hiervoor gebruiken we een eerste orde IIR filter:
		            \begin{itemize}[label={}]
			            \item \(y\left[ n \right] = x\left[ n \right] - x\left[ n - 1 \right] -0.999 \cdot y\left[ n - 1 \right]\)
		            \end{itemize}
	      \end{itemize}

	\item Framing and windowing
	      \begin{itemize}
		      \item Eerst delen de audio op in frames van 32 ms die elk 10 ms overlappen met elkaar
		      \item Daarna passen we een Hamming-venster toe om hoogfrequente ruis te vermeiden \cite{enwiki:windowing}
	      \end{itemize}

	\item Discrete Fourier Transform (DFT)
	      \begin{itemize}
		      \item Met de DFT transformeren we elke frame van het tijdsdomein naar het frequentiedomein
		      \item We padden de frames totdat ze een macht van 2 zijn. Zo kunnen we het Fast Fourier Transform (FFT) algoritme gebruiken. Dit is sneller en accurater dan de normale DFT als er omdat er bij de gewone DFT meer afrondingsfouten kunnen gebeuren. \cite{enwiki:FFT}
		      \item Het resultaat is een vector met de magnitudes van elke frequentie in het signaal. Die vector is zeer hoogdimensionaal en is niet optimaal voor geluidsherkenning.
	      \end{itemize}

	\item Mel-frequency warping
	      \begin{itemize}
		      \item Mel-frequency warping transformeert de originele hoogdimensionale frequentievector naar een Mel-frequency vector. Zie Figuur~\ref{fig:Mel-bands}.
		            \begin{itemize}[label={}]
			            \item \(f_{Mel} = 2595\log_{10}(1 + \frac{f}{700})\)
		            \end{itemize}
		      \item Die Mel-frequency vector bevat minder frequentiebanden en is daardoor veel lager in dimensies. De Mel-banden zijn echter zo verspreid zodat ze veel beter overeenkomen met de menselijke perceptie van geluid.
	      \end{itemize}

	\item Feature normalisation and scaling
	      \begin{itemize}
		      \item Tenslotte normaliseren we de logaritmes van de Mel-magnitudes zodat het gemiddelde nul is, en de standaardafwijking één is:
		            \begin{itemize}[label={}]
			            \item \(m'_b = \frac{\ln\left( m_b \right) - \mu_b}{\sigma_b}\)
		            \end{itemize}
		      \item Als allerlaatste pre-processingstap gaan we de data schalen zodat die zich in het interval \(\left[ -1,1 \right]\) bevind. Zo kunnen ze opgeslagen worden in Q0.7 formaat. (Zie hoofdstuk~\ref{section:Qm.f} op pagina~\pageref{section:Qm.f})
	      \end{itemize}
\end{enumerate}
\end{no-awa}


\begin{figure}[ht]
	\centering
	\includegraphics[width=0.6\textwidth]{figuren/Mel-frequency.png}
	\caption{Mel-frequency warping}
	\cite{efficient-feature-extraction}
	\label{fig:Mel-bands}
\end{figure}


\section{Convolutional Neural Network (CNN)}

Om met de Log-Mel samples geluidsevents te kunnen herkennen, gebruik ik een Convolutional Neural Network (CNN). Ons model bestaat uit meerdere verschillende lagen. De belangrijkste lagen zijn de convolutielagen. De eerste convolutielaag past een 2D-convolutie toe op het Log-Mel-spectrum met een bepaald aantal filters. Die filters gaan we optimaliseren met machine learning.
Na elke convolutielaag voegen we een ReLU activation laag toe. (niet weergegeven op figuur~\ref{fig:cnn-model}) De ReLU activatiefunctie is snel en gemakkelijk te berekenen, \cite{enwiki:relu} maar kan soms lijden tot een 'vanishing gradient'. \cite{enwiki:vanishing-gradient}

Om de dimensionaliteit van onze lagen kleiner te maken, introduceren wij een max pooling laag na elke activatielaag. Die laag neemt het maximum van een aantal waardes en laat de rest vallen. Daarna plaatsen wij een dropout laag. (niet weergegeven op figuur~\ref{fig:cnn-model}) Die gaat willekeurig, met een kans van 50 \%, de resultaten van verschillende convolutiefilters met nul vermenigvuldigen en dus laten vallen. Dit is om overfitting tijdens het trainen tegen te gaan.

\begin{figure}[ht]
	\centering
	\includegraphics[width=0.9\textwidth]{figuren/cnn.png}
	\caption{Voorbeeld van een CNN model}w
	\cite{slides:cnn}
	\label{fig:cnn-model}
\end{figure}

Die vier lagen (convolutie, activatie, max pooling en dropout) herhalen we drie keer. Daarna schakelen we over naar een traditioneel neuraal netwerk, ookwel 'fully connected netwerk' genoemd. Die bestaat uit fully connected layers of perceptronlagen, activatielagen en dropout lagen. Als allerlaatste outputlaag gebruiken wij een softmax activation layer om een genormaliseerde kansverdeling over de verschillende geluidsklassen. \cite{enwiki:softmax}

% TODO: Maybe add a table with the specific structure of the model and the amount of parameters

\subsection{Trainen van het model}
\subsubsection{Installatie Python en Tensorflow}
Het trainen van het model hebben we gewoon gedaan op onze computers in python. Hiervoor hebben wij de Keras uit de TensorFlow package gebruikt. Het installeren van alle benodigde packages ging echter niet vlekkeloos. Uit de TensorFlow documentatie konden we afleiden dat we specifiek Python 3.8 moesten installeren in plaats van de meest up-to-date versie van python. Dan bleek dat er nog meer packages nodig waren. Toch kon ik het relatief snel alle problemen oplossen.

% TODO: Decide if I leave this paragraph in or not
Voor Python gebruik ik Visual Studio Code, een all round text editor, om de code te schrijven en Powershell, een command line interface, om mijn Python omgeving manueel te managen. Ik vermoed dat ik, door die ervaring met Python projecten handmatig op te starten, gemakkelijker alle installatieproblemen kon vinden en oplossen.
Een requirements.txt file \cite{pip-user-guide} had bijvoorbeeld al een goede hulp geweest voor het installeren van de juiste packages.

\subsubsection{Resultaten}
Ik heb mijn model 400 epochs getrained in plaats van 250. Ik was namelijk vroeg klaar met al de rest en heb zo dus wat extra getained om te kunnen vergelijken. Omdat de model loss lager is bij 400 epochs dan bij 250 epochs, kunnen we afleiden dat het model waarschijnlijk beter is na 400 epochs. (zie figuur~\ref{fig:cnn-training})

\begin{figure}[ht]
	\centering
	\includegraphics[width=0.7\textwidth]{figuren/cnn_model.png}
	\caption{Loss van het model per epoch}
	\label{fig:cnn-training}
\end{figure}

In de confusion matrix (tabel~\ref{tab:confusion-matrix}) kunnen we zien dat het model het moeilijkste spraak kan detecteren.
Ik vermoed dat dat is omdat spraak het meest complexe spectrum zal geven. Achtergrondgeluid zal waarschijnlijk gewoon zacht geluid zijn, handgeklap is een luide impuls en fluiten is een relatief zuivere constante toon.
Al bij al is het model wel vrij correct.

\begin{table}[ht]
	\centering
	\begin{tabular}{ |c|c|c|c|c||c| }
		\cline{2-6}
		\multicolumn{1}{c| }{} & Predicted: & Predicted: & Predicted: & Predicted: &       \\
		\multicolumn{1}{c| }{} & background & handclap   & speech     & whistle    & Total \\

		\hline Actual:         &            &            &            &            &       \\
		background             & 25         & 0          & 0          & 0          & 25    \\

		\hline Actual:         &            &            &            &            &       \\
		handclap               & 1          & 23         & 1          & 0          & 25    \\

		\hline Actual:         &            &            &            &            &       \\
		speech                 & 0          & 3          & 22         & 0          & 25    \\

		\hline Actual:         &            &            &            &            &       \\
		whistle                & 0          & 0          & 0          & 25         & 25    \\
		\hline\hline
		                       &            &            &            &            &       \\
		Total                  & 26         & 26         & 23         & 25         & 100   \\
		\hline
	\end{tabular}
	\caption{Confusion matrix}
	\label{tab:confusion-matrix}
\end{table}

% TODO: Talk about F1 Score and such

\section{Edge implementatie}

\subsection{Kwantisatie van ons CNN-model}
\subsubsection{Qm.f formaat}
\label{section:Qm.f}

\subsection{Experiment met microfoon}
\subsubsection{Resultaten}


\section{Conclusie}
% \section{Inleiding}

Communicatie met een ruimtetuig (voor bijvoorbeeld een Marsmissie) is cruciaal voor de goede werking van een verblijf in de ruimte om verschillende redenen. Enerzijds superviseert het controlecentrum op aarde de werking van de missie aan de hand van de resultaten van proeven, foto’s, meetwaarden, ... en anderzijds geeft het opdrachten door aan de bemanning en bestuurt het ruimtetuig.

De enige manier om te communiceren met een toestel in de ruimte zijn radiogolven.

Dit rapport gaat over een aantal principes om deze communicatie te realiseren. In een eerste deel gaat het over een beschrijving van het ‘Deep Space Network’ met het doel en de werking van de up- en downlink. In een tweede deel gaat het dieper in op fundamentelere onderwerpen zoals allereerst “Wat zijn radiogolven en hoe bevatten deze de informatie?”, vervolgens op “Hoe werken schotelantennes?” om te eindigen met “Waarom is communicatie met ruimtetuigen op grote afstand moeilijk?”.

\section{Communicatie met een ruimtemissie}

\subsection{Wat is het 'Deep Space Network'?}

Het ‘Deep Space Network’ (DSN), dat sinds 1958 als een onderdeel van NASA (National Aeronautics and Space Administration) bestaat, is een internationaal netwerk dat door middel van drie gigantische antennes op de aarde communicatie mogelijk maakt met satellieten of ruimtetuigen \cite{martin}. Het eerste deep-space communicatiecentrum is te vinden in de Mojave-woestijn in Californië nabij Goldstone, een tweede nabij Madrid in Spanje en een derde in Canberra, Australië. Door deze geografische spreiding, telkens ongeveer 120° verder op de aardbol, is steeds communicatie mogelijk met om het even waar in de ruimte \cite{christiaens}.

\subsection{Wat is de uplink en de downlink bij een 'Deep Space Network'?}

De uplink is het verzenden van informatie vanop aarde naar een ruimtetuig. De uplink doet een upload. De downlink gaat in de andere richting: van een ruimtetuig naar de aarde. Deze doet een download. De vertraging is de tijd die verstrijkt tussen het verzenden en ontvangen van een signaal. Radiogolven zijn, net zoals licht, elektromagnetische golven. De voortplantingssnelheid is voor radiogolven 3.108 meter per seconde in lucht of het luchtledige. Dit lijkt snel, maar is eigenlijk (relatief) traag wegens de enorme te overbruggen afstanden \cite{updown}. Deze afstand is bijvoorbeeld gemiddeld 384 450 km tot de maan \cite{maan} en 230 000 000 km tot Mars \cite{mars}.

\subsection{Wat zijn radiogolven en hoe bevatten deze informatie?}

Elektromagnetische golven zijn, op basis van hun frequentie of golflengte, onder te verdelen in radiogolven, microgolven, lichtstralen, röntgenstralen, gammastraling, … \cite{radiowaves}, zoals te zien in figuur \ref{fig:EMS} op bladzijde \pageref{fig:EMS}.

\begin{figure}[ht]
  \centering
  \includegraphics[width=0.5\textwidth]{voorbeeld_figuren/spectrum_straling}
  \caption{Elektromagnetische straling.}
  \cite{energiezonnepanelen} %Als je deze cite in de caption plaatst, komt de referentie ook in de lijst van figuren voor en is de volgorde niet juist.
  \label{fig:EMS}
\end{figure}

De frequentie f van radiogolven is het aantal keer per seconde dat het signaal maximaal wordt. De eenheid van frequentie is Hz (Hertz). De golflengte is de afstand tussen één maximum en het volgende maximum van het signaal. De elektrische en magnetische component en voortplantingsrichting zijn te zien in figuur \ref{fig:golf} op bladzijde \pageref{fig:golf}.

\begin{figure}[ht]
  \centering
  \includegraphics[width=0.5\textwidth]{voorbeeld_figuren/elektromagnetische_golf}
  \caption{Voortplanting elektromagnetische golf.}
  \cite{elektromagnetischestraling}
  \label{fig:golf}
\end{figure}

Een zeer belangrijk verband tussen de frequentie f en de golflengte $\lambda$ is: $v = f\lambda$, met v de voortplantingssnelheid van de golf \cite{elektromagnetischestraling}. Bij een grotere golflengte zijn de golven immers breder en duurt het dus langer eer een volledige golf ergens voorbij komt (lagere frequentie). Deze tijd is de periode T. Er geldt: $T = \frac{1}{f}$. De golflengte van zichtbaar licht is (veel) kleiner dan van radiogolven: ordegrootte $10^{-6}$ meter.

Radiogolven voor radio en TV hebben dan weer een grotere golflengte dan radiogolven voor communicatie in de ruimte. Bij deze laatste  is de golflengte zeer klein: ongeveer één centimeter.

\begin{figure}[ht]
  \centering
  \includegraphics[width=0.5\textwidth]{voorbeeld_figuren/radiospectrum}
  \caption{Radiospectrum.}
  \cite{wavelengths}
  \label{fig:radiospectrum}
\end{figure}

Een radiogolf voldoet aan de uitdrukking:
\begin{equation*} %eventueel \label{} bijvoegen om te kunnen refereren.
v = A\sin(\omega t+\phi)
\end{equation*}	%equation zonder * geeft een nummer bij de uitdrukking. Dit is nodig als je naar deze formule wil refereren.
Dit impliceert dat een golf informatie kan bevatten door ofwel de amplitude A ofwel de hoek $(\omega t+\phi)$ ofwel eventueel een combinatie van amplitude en hoek te veranderen. Bij amplitudemodulatie (AM) bevat de amplitude de informatie en bij frequentiemodulatie (FM) is dat $\omega$ \cite{radiowaves1}.  Hierbij is $\omega = 2\pi f$. Een variante op frequentiemodulatie is fasemodulatie (PM). Bij fasemodulatie is de fase een functie van het door te sturen signaal.

\subsubsection{Amplitudemodulatie}

Bij amplitudemodulatie \cite{kennedy} (AM) is de golfvorm het resultaat van een hoogfrequent signaal, de draaggolf met een vaste frequentie, waarvan de amplitude veranderlijk is naargelang de door te zenden informatie. Een positief (audio)signaal levert een iets hogere draaggolfamplitude, een negatief signaal levert een iets lagere draaggolfamplitude. Wiskundig is de amplitude van de draaggolf gelijk aan de som van een constant signaal (gelijkspanning) met het door te zenden signaal. De mate waarin de draaggolf het bronsignaal volgt, heet de modulatiediepte, uitgedrukt in procenten van de ongemoduleerde draaggolf. Deze is steeds kleiner dan 1. Figuur \ref{fig:AM} op bladzijde \pageref{fig:AM} toont een informatiesignaal, de ongemoduleerde draaggolf (carrier in het Engels) en ten slotte het resulterend signaal.

\begin{figure}[ht]
  \centering
  \includegraphics[width=0.5\textwidth]{voorbeeld_figuren/AM}
  \caption{AM modulatie.}
  \cite{AM}
  \label{fig:AM}
\end{figure}

\subsubsection{Frequentiemodulatie}

Bij frequentiemodulatie \cite{kennedy} (FM) is de bekomen golfvorm opnieuw het resultaat van een hoogfrequent signaal, de draaggolf, maar waarbij nu de amplitude constant is en de frequentie veranderlijk naargelang de door te zenden informatie. Hoe hoger de amplitude van het door te zenden signaal, hoe hoger de frequentie van de aangepaste draaggolf. Frequentiemodulatie is minder gevoelig aan ruis dan amplitudemodulatie. Daar staat wel tegenover dat een FM-signaal bij een gegeven bronsignaal in het algemeen, afhankelijk van de modulatie-index, veel meer bandbreedte nodig heeft dan een AM-signaal. Figuur \ref{fig:FM} op bladzijde \pageref{fig:FM} toont een informatiesignaal, de ongemoduleerde draaggolf (carrier in het Engels) en ten slotte het resulterend signaal (de gemoduleerde draaggolf).

\begin{figure}[ht]
  \centering
  \includegraphics[width=0.5\textwidth]{voorbeeld_figuren/FMwit}
  \caption{FM modulatie.}
  \cite{FM}
  \label{fig:FM}
\end{figure}


\subsection{Hoe werken schotelantennes?}

Elk type antenne heeft zijn toepassingsgebied. Voor communicatie met een ruimtetuig, zijn alleen schotelantennes bruikbaar \cite{kennedy}. De reden hiervoor is dat alleen schotelantennes hun energie in één welbepaalde enge bundel kunnen verzenden.

\begin{figure}[ht]
  \centering
  \includegraphics[width=0.3\textwidth]{voorbeeld_figuren/paraboolantenne_doorsnede}
  \caption{Doorsnede van een paraboolantenne.}
  \cite{kennedy}
  \label{fig:parabool}
\end{figure}

Schotelantennes hebben een parabolische vorm, zoals te zien in figuur \ref{fig:parabool} op bladzijde \pageref{fig:parabool}. Een eigenschap van deze vorm is dat alle signaalenergie die erop invalt, wordt geconcentreerd in het brandpunt (focus) en omgekeerd dat alle energie die vertrekt vanuit het brandpunt naar de parabool, verder reist in een zeer gericht signaal.

Kenmerkend voor schotelantennes is dat ze werken bij hoge frequenties en bijgevolg kleine golflengten.

Een antenne heeft als doel zoveel mogelijk het signaal in één richting te sturen. Dit lukt beter naarmate de golflengte kleiner en de antenne groter is. Communicatie in de ruimte vereist bijgevolg kleine golflengten en dus hoge frequenties. De antennes voor het Deep Space Network zijn bovendien zeer groot!

\subsection{Waarom is communicatie met ruimtetuigen op grote afstand moeilijker?}

De oppervlakte van een bol is gelijk aan $4\pi r^{2}$. Het ontvangen signaal op een bepaalde afstand r van de antenne is bijgevolg omgekeerd evenredig met $r^{2}$ of dus evenredig met $r^{-2}$. Wanneer de afstand dus verdubbelt, verkleint het vermogen van het signaal al vier keer!

In het geval van communicatie met een ruimtetuig is de afstand zeer groot. Het ontvangen signaal is bijgevolg zeer klein en dus moet de schotelantenne zeer groot zijn om toch maar voldoende invallend vermogen te kunnen verzamelen in het brandpunt \cite{radiocontact}. Een voorbeeld hiervan is figuur \ref{fig:effelsberg} op bladzijde \pageref{fig:effelsberg}.

\begin{figure}[ht]
  \centering
  \includegraphics[width=0.6\textwidth]{voorbeeld_figuren/paraboolantenne}
  \caption{Effelsberg telescoop.}
  \cite{effelsberg}
  \label{fig:effelsberg}
\end{figure}

\section{Geplande ruimtevluchten}

Een overzicht van toekomstige ruimtemissies naar de maan kan je vinden in tabel \ref{tab:missies} op bladzijde \pageref{tab:missies}.

\begin{table}[ht]
\centering
\small
\begin{tabular}{cc|c} %De | duidt aan waar je een verticale lijn in de tabel wil.
Land &Jaar &Missie \\ \hline
India &2018 &Chandrayaan-2 \\
Verenigde Staten &2018 &Lunar Flashlight \\
Verenigde Staten &2018 &Lunar Ice Cube \\
Japan &2019 &SLIM \\
Zuid-Korea &2020 &Moon orbiter \\
Verenigd Koninkrijk &2024 &Lunar Mission One \\
\end{tabular}
\caption{Land, jaartal en naam van enkele toekomstige missies.}
\cite{missies}
\label{tab:missies}
\end{table}

\section{Besluit}

Communicatie met een missie naar Mars is zeker mogelijk dankzij de zeer grote schotelantennes van het ‘Deep Space Network’. Deze antennes kunnen permanent voldoende sterke signalen uit de ruimte ontvangen of naar de ruimte versturen dankzij hun grote afmetingen, hun geografische spreiding en de gebruikte frequenties. Amplitudemodulatie of frequentiemodulatie zijn bruikbare technieken om de signalen uit te zenden of te ontvangen.
%\end{multicols}

\scriptsize
\newpage
\begin{flushleft}
\bibliographystyle{IEEEtran}
\bibliography{bronnen} %Je mag deze file een andere naam geven, maar de extensie moet 'bib' zijn.
\end{flushleft}

\end{document}